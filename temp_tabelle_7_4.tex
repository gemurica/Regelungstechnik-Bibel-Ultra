\documentclass[12pt]{article}
\usepackage[T1]{fontenc}
\usepackage[utf8]{inputenc}
\usepackage[ngerman]{babel}
\usepackage{array}
\usepackage{hhline}
\usepackage{multirow}
\usepackage{graphicx}
\usepackage{amsmath}

\begin{document}

\begin{table}[ht]
\centering
\setlength{\tabcolsep}{5pt}

\resizebox{\textwidth}{!}{%
\begin{tabular}{||>{\centering\arraybackslash}p{7mm}||
c|
c|c|c|c||
c|c|c|c||}

% ==========================================
% Doppelter oberer Rahmen
\hhline{|t:==:t:==:t|}
% ==========================================

% ===== Kopfzeile 1 =====
\multicolumn{2}{||c||}{} &
\multicolumn{4}{c||}{Korrekturwerte f\"ur Betrag} &
\multicolumn{4}{c||}{Korrekturwerte f\"ur Phasenwinkel} \\

% Nur unter den 8 rechten Spalten Linie
\hhline{||~~|----||----||}

% ===== Kopfzeile 2 =====
\multicolumn{2}{||c||}{} &
\multicolumn{4}{c||}{$\lg(|G|)-\lg(|G|_{\text{Asymptote}})$} &
\multicolumn{4}{c||}{$|\varphi-\varphi_{\text{Asymptote}}|$} \\

\hhline{||~~|----||----||}

% ===== Kopfzeile 3 =====
 & $D$ &
$\omega/\omega_0=0.5$ & $1$ & $2$ & $5$ &
$\omega/\omega_0=0.5$ & $1$ & $2$ & $5$ \\

% Doppelte Trennlinie unter Kopf
\hhline{|t:==:t:==:t|}

% ======================================================
% ===================== PT1 ============================
% ======================================================

\multirow{4}{*}{\rotatebox{90}{PT1}} 
& 0.1 & 0 & 0 & 0 & 0 & 0 & 0 & 0 & 0 \\

% Linie nur ab Spalte 2
\hhline{||~|----||----||}

& 0.3 & 0 & 0 & 0 & 0 & 0 & 0 & 0 & 0 \\
\hhline{||~|----||----||}

& 0.5 & 0 & 0 & 0 & 0 & 0 & 0 & 0 & 0 \\
\hhline{||~|----||----||}

& 1.0 & 0 & 0 & 0 & 0 & 0 & 0 & 0 & 0 \\

% Doppeltrennung zwischen PT1 und PT2
\hhline{|t:==:t:==:t|}

% ======================================================
% ===================== PT2 ============================
% ======================================================

\multirow{5}{*}{\rotatebox{90}{PT2}} 
& 0.0 & 0 & 0 & 0 & 0 & 0 & 0 & 0 & 0 \\

\hhline{||~|----||----||}

& 0.2 & 0 & 0 & 0 & 0 & 0 & 0 & 0 & 0 \\
\hhline{||~|----||----||}

& 0.4 & 0 & 0 & 0 & 0 & 0 & 0 & 0 & 0 \\
\hhline{||~|----||----||}

& 0.7 & 0 & 0 & 0 & 0 & 0 & 0 & 0 & 0 \\
\hhline{||~|----||----||}

& 1.0 & 0 & 0 & 0 & 0 & 0 & 0 & 0 & 0 \\

% Unterer doppelter Rahmen
\hhline{|b:==:b:==:b|}

\end{tabular}
}
\caption{Tabelle 7-4 (Struktur mit hhline exakt kontrolliert)}
\end{table}

\end{document}
